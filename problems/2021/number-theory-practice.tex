\documentclass[a4paper,10pt,landscape,twocolumn]{scrartcl}

%% Settings
\newcommand\problemset{: Computational Number Theory}
\newif\ifcomments
\commentsfalse % hide comments
%\commentstrue % show comments

%% Packages
\usepackage[english]{exercises}
\usepackage{wasysym}
\usepackage{hyperref}
\hypersetup{colorlinks=true, urlcolor = blue, linkcolor = blue}
\usepackage{enumitem}
\usepackage{graphicx}

%% Macros
\usepackage{xspace}

\newcommand{\eps}{\varepsilon}
\newcommand{\ket}[1]{|#1\rangle}
\newcommand{\bra}[1]{\langle#1|}
\newcommand{\inp}[2]{\langle{#1}|{#2}\rangle}
\newcommand{\norm}[1]{\parallel\!#1\!\parallel}
\newcommand{\points}[1]{\marginpar{\textbb{#1 p.}}}
\newtheorem{theorem}{Theorem}
\newtheorem{definition}{Definition}
\newtheorem{proposition}{Proposition}
%\newenvironment{proof}{\noindent {\bf Proof }}{{\hfill $\Box$}\\}

\newcommand{\gen}{\ensuremath{\mathsf{Gen}}\xspace}
\newcommand{\enc}{\ensuremath{\mathsf{Enc}}\xspace}
\newcommand{\dec}{\ensuremath{\mathsf{Dec}}\xspace}
\newcommand{\mac}{\ensuremath{\mathsf{Mac}}\xspace}
\newcommand{\vrfy}{\ensuremath{\mathsf{Vrfy}}\xspace}
\newcommand{\negl}{\ensuremath{\mathsf{negl}}\xspace}
\newcommand{\PrivK}{\ensuremath{\mathsf{PrivK}}\xspace}
\newcommand{\eav}{\ensuremath{\mathsf{eav}}\xspace}

\newcommand{\A}{\ensuremath{\mathcal{A}}}

\newcommand{\Z}{\ensuremath{\mathbb{Z}}}
\newcommand{\R}{\ensuremath{\mathbb{R}}}
\newcommand{\N}{\ensuremath{\mathbb{N}}}


\newcommand\floor[1]{\lfloor#1\rfloor}
\newcommand\ceil[1]{\lceil#1\rceil}

% \newcommand{\comment}[1]{{\sf [#1]}\marginpar[\hfill !!!]{!!!}}
\newcommand{\chris}[1]{\comment{\color{blue}Chris: #1}}
\newcommand{\jan}[1]{\comment{\color{magenta}Jan: #1}}




\begin{document}

\practiceproblems

As during the written exam, all these problems should be solved by hand, without the help of electronic devices (except for double-checking your solutions).

\begin{exercise}[generating elements]

\begin{subex}
  Show that 5 is a generator of $\mathbb{Z}_7^*$.
\end{subex}
\begin{subex}
  Show that 4 generates a subgroup of size 3 of $\mathbb{Z}_7^*$.
\end{subex}
\begin{subex}
  Show that 3 is a generator of $\mathbb{Z}_{17}^*$.
\end{subex}

\textbf{Hint: } In these problems, it can save you some computation power, if you work with negative numbers.
 For example observe that when computing modulo $17$, it holds that $15 = (-2)$.
  Hence, rather than computing $15 \cdot 3=45=2 \cdot 17+11=11 \mod 17$, it is quicker to compute $(-2) \cdot 3=(-6)=11 \mod 17$.
\end{exercise}

\begin{exercise} [square-and-multiply] Use square-and-multiply to compute the following. Don't forget to reduce all numbers $\mod N$ on the way to simplify the calculations!
\begin{subex}
$[3^{65} \mod 7]$
\end{subex}
\begin{subex}
$[7^3 \mod 10]$
\end{subex}
\begin{subex}
$[7^{131} \mod 10]$
\end{subex}
\begin{subex}
$[5^{65} \mod 21]$
\end{subex}
\textbf{Hint: } For this type of problems, Fermat's little theorem often provides you some nice shortcuts.
\end{exercise}

\begin{exercise} [greates common divisors] Use the Euclidean algorithm
  to compute
\begin{subex}
  $\gcd(14,91)$
\end{subex}
\begin{subex}
  $\gcd(126, 399)$
\end{subex}
\begin{subex}
  $\gcd(126,400)$
\end{subex}
\end{exercise}

\begin{exercise} [multiplicative inverses] Use the extended Euclidean
  algorithm to compute
\begin{subex}
integers $a,b \in \mathbb{Z}$ such that $a \cdot 91 + b \cdot 14 = \gcd(91,14)$
\end{subex}
\begin{subex}
integers $a,b \in \mathbb{Z}$ such that $a \cdot 45 + b \cdot 16 = 1$
\end{subex}
\begin{subex}
$[16^{-1} \mod 45]$
\end{subex}
\begin{subex}
$[13^{-1} \mod 16]$
\end{subex}
\begin{subex}
$[7^{-1} \mod 9]$
\end{subex}
\end{exercise}
\end{document}
