\documentclass[a4paper,10pt]{article}
\usepackage{solutions}
\usepackage{amsmath,amsthm,amssymb,amsfonts, enumitem, fancyhdr,
  color, comment, graphicx, environ}

\begin{document}
\frontpage{Your team name}{Your names}{2}

%%%%%%%%%%%%%%%%%%%%%%%%%%%%%%%%%%%%%%%%%%%%%%%
\begin{groupproject}{8}
Read up on what a
\href{https://en.wikipedia.org/wiki/Fermi_problem}{Fermi estimate}
is.
\begin{enumerate}
\item Give a Fermi estimate for all passwords used in the world.
\item Assume all passwords used in the world are distinct. How long do
  most of them have to be?
\end{enumerate}

Post your results to \href{https://canvas.uva.nl/courses/10514/discussion_topics/76834?module_item_id=282355}{this discussion forum}. Remember that when doing Fermi estimates, it is important to state clearly what assumptions you make when giving your answer.
\end{groupproject}


%%%%%%%%%%%%%%%%%%%%%%%%%%%%%%%%%%%%%%%%%%%%%%%
\begin{nquestion}{1}{2 messages with the same key}
Two ASCII messages containing English letters and spaces only are encrypted using the one-time pad and the same key. The 8th byte of the first ciphertext is observed to be $\mathtt{0xAA}$ and the 8th byte of the second ciphertext is observed to be $\mathtt{0xE8}$. Let $m_1$ (resp., $m_2$) denote the 8th ASCII character in the first (resp., second) message. What is the most you can conclude about $m_1$ and $m_2$?
\begin{enumerate}
\item Nothing can be determined about $m_1$ or $m_2$ since the one-time pad is perfectly secret.
\item $m_1$ is the character \texttt{A} and $m_2$ is the character \texttt{D}.
\item One of $m_1$ or $_2$ is the space character, and the other is the character \texttt{b}.
\item $m_1$ is the character \texttt{b} and $m_2$ is the space character.
\item $m_1$ is the space character and $m_2$ is the character \texttt{b}.
\end{enumerate}
\end{nquestion}

\begin{solution}
  Your solution here. Answers have to be fully justified in order to get the full amount of points.
\end{solution}

%%%%%%%%%%%%%%%%%%%%%%%%%%%%%%%%%%%%%%%%%%%%%%%
\begin{nquestion}{2}{3 messages with the same key}
Three ASCII messages containing English letters and spaces only are encrypted using the one-time pad and the same key. The 10th byte of the first ciphertext is observed to be \texttt{0x66}, the 10th byte of the second ciphertext is observed to be \texttt{0x32}, and the 10th byte of the third ciphertext is observed to be \texttt{0x23}. Let $m_1$ (resp., $m_2$, $m_3$) denote the 10th ASCII character in the first (resp., second, third) message. Explain how to determine $m_1, m_2$ and $m_3$. (1/3 points for the values they obtain, 1 for the correct derivation).
\end{nquestion}
\begin{solution}
  Your solution here
\end{solution}

%%%%%%%%%%%%%%%%%%%%%%%%%%%%%%%%%%%%%%%%%%%%%%%
\begin{question}{1}
Which of the following is a negligible function?

\begin{enumerate}
\item $f(n) = \frac{1}{2^n}$
\item $f(n) = \frac{n}{2^n}$
\item $f(n) = \frac{1}{n}$
\item $f(n) = \frac{1}{2}$
\end{enumerate}
\end{question}

\begin{solution}
justify your answers
\end{solution}

%%%%%%%%%%%%%%%%%%%%%%%%%%%%%%%%%%%%%%%%%%%%%%%
\begin{nquestion}{3}{Non-equal-length messages, Exercise 3.3 from [KL]}
Say $\Pi=(\gen, \enc, \dec)$ is such that for $k \in \set{0,1}^n$, algorithm $\enc_k$ is only defined for messages of length at most $\ell(n)$ (for some polynomial $\ell$). Assuming $\enc_k$ is EAV-secure, construct a scheme $\Pi'$ satisfying Definition 3.8 even when the adversary is \emph{not} restricted to outputting equal-length messages in $\PrivK^{\eav}_{\A, \Pi}$.
\end{nquestion}
\begin{solution}
\end{solution}

%%%%%%%%%%%%%%%%%%%%%%%%%%%%%%%%%%%%%%%%%%%%%%%
\begin{nquestion}{2}{Success probability in a reduction}
Assume that algorithm $A$ efficiently solves problem $P_1$ and there is an efficient reduction from problem $P_2$ to problem $P_1$. Prove that there exists an algorithm $B$ for which
\[
\Pr[A \text{ succeeds in solving $P_1$}] \leq \Pr[B \text{ succeeds in solving $P_2$}] \, .
\]
\end{nquestion}

\begin{solution}
\end{solution}


%%%%%%%%%%%%%%%%%%%%%%%%%%%%%%%%%%%%%%%%%%%%%%%
\begin{nquestion}{1}{Pseudo One-Time Pad using a PRG}
Let $G$ be a function mapping $n$-bit inputs to $2n$-bit outputs. Which of the following is true of the pseudo one-time pad encryption scheme based on $G$? (Check all that apply.)
\begin{enumerate}
\item The scheme can be used to securely encrypt multiple messages using the same key.
\item The scheme is perfectly secret.
\item The key space of the scheme is at least as large as the message space.
\item The scheme is computationally secret if $G$ is a pseudorandom generator.
\end{enumerate}
\end{nquestion}

\begin{solution}
\end{solution}

%%%%%%%%%%%%%%%%%%%%%%%%%%%%%%%%%%%%%%%%%%%%%%%
\begin{question}{3} \label{hardquestion}
% \newcounter{hardquestion}
% \setcounter{hardquestion}{\value{questionnr}}
Let $n$ be even and let $G$ be a pseudorandom generator with expansion factor $\ell(n)=4n$. In this exercise, we want to show that $G'(s) := G(s_1, \ldots, s_{\ceil{n/2}})$, where $s=s_1 \cdots s_n$, is a PRG.

Let $D$ be an efficient distinguisher for $G'$ with distinguishing advantage
\[
\eps'(n) := | \Pr_{s \leftarrow \set{0,1}^n}[D(G'(s))=1] - \Pr_{r \leftarrow \set{0,1}^{2n}}[D(r)=1]|
\]
Let us define the distinguishing advantage of $D$ for $G$ as
\[
\eps(n) := |\Pr_{s \leftarrow \set{0,1}^n}[D(G(s))=1]- \Pr_{r \leftarrow \set{0,1}^{4n}}[D(r)=1]| \, .
\]
Now, argue (1pt) that $\eps'(n)=\eps(n/2)$. Then use this derivation to conclude that $G'$ is a PRG (2pts).
\end{question}
\begin{solution}
\end{solution}

%%%%%%%%%%%%%%%%%%%%%%%%%%%%%%%%%%%%%%%%%%%%%%%
\begin{question}{2}
Let $n$ be even and let $G$  be a pseudorandom generator with expansion factor $\ell(n)=4n$.  Define $G'(s):=G(0^{|s|} \| s)$. Prove that $G'$ is not a PRG.

\noindent \textbf{Hint:} use the Question~\the\numexpr\value{questionnr}-1\relax\ above.
\end{question}
\begin{solution}
\end{solution}

%%%%%%%%%%%%%%%%%%%%%%%%%%%%%%%%%%%%%%%%%%%%%%%
\begin{question}{2}
Let $n$ be even and let $G$ be a pseudorandom generator with expansion factor $\ell(n)=4n$.
Define $G'(s):=G(s) \| G(s \oplus 0^{n-1}1)$. Show that $G'$ is not a PRG.

\noindent\textbf{Hint:} use the Question~\the\numexpr\value{questionnr}-2\relax\ above.
\end{question}
\begin{solution}
\end{solution}

%%%%%%%%%%%%%%%%%%%%%%%%%%%%%%%%%%%%%%%%%%%%%%%
\begin{question}{4}
Let $G:\set{0,1}^n \rightarrow \set{0,1}^{2n}$ be a PRG. Describe a computationally unbounded adversary $\A$ that distinguishes the output of $G$ from a uniform $2n$-bit string with probability exponentially close to 1. How does it work? Compute its exact distinguishing advantage.

\noindent\textbf{Hint:} Use practice problem 4.

What is $\Pr[PRG_{\A,G}(n) =1]$ for this adversary (where $PRG_{\A,G}$ is the experiment defined in Figure 2 of problem set 2)?
\end{question}
\begin{solution}
\end{solution}



%%%%%%%%%%%%%%%%%%%%%%%%%%%%%%%%%%%%%%%%%%%%%%%
\begin{programming}{8}
Solve \href{https://www.coursera.org/learn/cryptography/supplement/3Iajv/programming-assignment-2}{programming assignment 2}. Describe the solution and how you proceeded. You may use any programming language you like to solve the problem.
In \href{https://github.com/cschaffner/ModernCryptographyCourse/tree/master/programming/MultiTimePad}{this github repo}, you can find some possibly helpful python code. Use your program to solve \href{https://github.com/cschaffner/ModernCryptographyCourse/blob/master/programming/MultiTimePad/ciphertexts.txt}{this same type of problem} as well.
\end{programming}

\begin{solution}
\end{solution}


\end{document}
