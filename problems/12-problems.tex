\documentclass[a4paper,10pt,landscape,twocolumn]{scrartcl}

%% Settings
\newcommand\problemset{12}
\newcommand\worksession{Friday, 20 October 2017}
\newif\ifcomments
\commentsfalse % hide comments
%\commentstrue % show comments

%% Packages
\usepackage[english]{exercises}
\usepackage{wasysym}
\usepackage{hyperref}
\hypersetup{colorlinks=true, urlcolor = blue, linkcolor = blue}
\usepackage{enumitem}

%% Macros
\usepackage{xspace}

\newcommand{\eps}{\varepsilon}
\newcommand{\ket}[1]{|#1\rangle}
\newcommand{\bra}[1]{\langle#1|}
\newcommand{\inp}[2]{\langle{#1}|{#2}\rangle}
\newcommand{\norm}[1]{\parallel\!#1\!\parallel}
\newcommand{\points}[1]{\marginpar{\textbb{#1 p.}}}
\newtheorem{theorem}{Theorem}
\newtheorem{definition}{Definition}
\newtheorem{proposition}{Proposition}
%\newenvironment{proof}{\noindent {\bf Proof }}{{\hfill $\Box$}\\}

\newcommand{\gen}{\ensuremath{\mathsf{Gen}}\xspace}
\newcommand{\enc}{\ensuremath{\mathsf{Enc}}\xspace}
\newcommand{\dec}{\ensuremath{\mathsf{Dec}}\xspace}
\newcommand{\mac}{\ensuremath{\mathsf{Mac}}\xspace}
\newcommand{\vrfy}{\ensuremath{\mathsf{Vrfy}}\xspace}
\newcommand{\negl}{\ensuremath{\mathsf{negl}}\xspace}
\newcommand{\PrivK}{\ensuremath{\mathsf{PrivK}}\xspace}
\newcommand{\eav}{\ensuremath{\mathsf{eav}}\xspace}

\newcommand{\A}{\ensuremath{\mathcal{A}}}

\newcommand{\Z}{\ensuremath{\mathbb{Z}}}
\newcommand{\R}{\ensuremath{\mathbb{R}}}
\newcommand{\N}{\ensuremath{\mathbb{N}}}


\newcommand\floor[1]{\lfloor#1\rfloor}
\newcommand\ceil[1]{\lceil#1\rceil}

% \newcommand{\comment}[1]{{\sf [#1]}\marginpar[\hfill !!!]{!!!}}
\newcommand{\chris}[1]{\comment{\color{blue}Chris: #1}}
\newcommand{\jan}[1]{\comment{\color{magenta}Jan: #1}}




\begin{document}

\problems

{\sffamily\noindent
We will work on the following exercises together during the work sessions on \worksession.

You are strongly encouraged to work together on the exercises, including the homework. You do not have to hand in solutions to these problem sets.}


\begin{exercise}[Insecurity of plain RSA Signatures]

In Section 12.4.1 we showed an attack on the
plain RSA signature scheme in which an attacker forges a signature
on an arbitrary message using two signing queries. Show how an
attacker can forge a signature on an arbitrary message using a
\emph{single} signing query.

\textbf{Hint:} What is the signature of $m=\tilde{m}^e$ for some $\tilde{m}$?
\end{exercise}

\begin{exercise}[One-time secure signature scheme?]
	
Let $f$ be a one-way permutation. Consider the following signature
scheme for messages in the set $\{1,\cdot , n\}$:
\begin{itemize}
\item To generate keys, choose uniform $x \in \{0, 1\}^n$ and set $y := f (n) (x)$
(where $f (i) (·)$ refers to $i$-fold iteration of $f$ , and $f^{(0)} (x) = x$). The
public key is $y$ and the private key is $x$.
\item To sign message $i \in \{1,\dots , n\}$, output $f^{(n−i)} (x)$.
\item To verify signature $\sigma$ on message $i$ with respect to public key $y$,
check whether $y = f^{(i)} (\sigma)$.
\end{itemize}

\begin{enumerate}
\item Show that the above is not a one-time-secure signature scheme.
Given a signature on a message $i$, for what messages $j$ can an
adversary output a forgery?
\item Prove that no ppt adversary given a signature on $i$ can output a
forgery on any message $j > i$ except with negligible probability.
\item Suggest how to modify the scheme so as to obtain a one-time-secure
signature scheme.

\textbf{Hint}: Include two values $y$, $y'$ in the public key.	
\end{enumerate}

	
\end{exercise}

\begin{exercise}

Let $f$ be a permutation and $f^i(x)$ the $i$-fold iteration of $f$,
and $f^{(0)}(x) := x$. Let us consider the following signature scheme
$\Pi = (\mathsf{Gen},\mathsf{Sign},\mathsf{Vrfy})$ for messages $m \in \{1, \ldots, p\}$ with $p = p(n)$ polynomial in $n$.
\begin{center}\vspace{-1em}
\begin{tabular}{rcl}
  $\mathsf{Gen}(1^n)$ & : &
    Choose $\mathsf{sk}_1,\mathsf{sk}_2 \in_R \{0,1\}^n$, $\mathsf{pk}_1 := f^p(\mathsf{sk}_1)$ and $\mathsf{pk}_2 := f^p(\mathsf{sk}_2)$. \\
    & & Set $\mathsf{sk} := (\mathsf{sk}_1,\mathsf{sk}_2)$ and $\mathsf{pk} := (\mathsf{pk}_1,\mathsf{pk}_2)$.\\[5pt]
  $\mathsf{Sign}_{\mathsf{sk}}(m)$ & : &
    Compute $\sigma_1 := f^{p-m}(\mathsf{sk_1})$ and $\sigma_2 := f^{m-1}(\mathsf{sk}_2)$. Return $\sigma := (\sigma_1,\sigma_2)$.\\[5pt]
  $\mathsf{Vrfy}_{\mathsf{pk}}(m,\sigma)$ & : &
    If $\mathsf{pk}_1 =
  f^m(\sigma_1)$ and $\mathsf{pk}_2 = f^{p-m+1}(\sigma_2)$ return $1$,
  else return $0$.
\end{tabular}
\end{center}
\begin{enumerate}
\item Show that $\Pi$ is correct.
\item Prove that $\Pi$ is a one-time-secure signature scheme, if
  $f$ is a one-way permutation.
\end{enumerate}



\end{exercise}


\end{document}
